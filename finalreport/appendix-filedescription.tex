\section{Description of files}
The following files are included in the final project directory:

\begin{itemize}
\item The Logic Simulator program

  The core files of the program, required for the bare minimum functionality.
  See individual reports for details of the classes provided.
  \begin{itemize}
  \item \texttt{logsim.py}: The main program to be launched by the user.
  \item \texttt{names.py}: Provides the \texttt{Names} class.
  \item \texttt{scanner.py}: Provides the \texttt{Scanner} and
    \texttt{Symbol} classes.
  \item \texttt{parse.py}: Provides the \texttt{Parser} class.
  \item \texttt{devices.py}: Provides the \texttt{Device} and
    \texttt{Devices} classes.
  \item \texttt{network.py}: Provides the \texttt{Network} class.
  \item \texttt{monitors.py}: Provides the \texttt{Monitors} class.
  \item \texttt{userint.py}: Provides the \texttt{UserInterface} class,
    that is, the command-line interface.
  \item \texttt{gui.py}: Provides the \texttt{Gui} class and the
    \texttt{MyGLCanvas} class which is used in the GUI.
  \end{itemize}
\item Translation files

  These are the files needed to provide translations of the GUI to
  other languages. The structure of these files follows conventions
  of GNU gettext.

  \begin{itemize}
  \item \texttt{locale/}: Directory containing all translation files.
  \item \texttt{locale/LogicSimulator.pot}: Portable Object Template
    file containing all the strings to be translated.
  \item \texttt{locale/<locale>/LC\_MESSAGES/LangDomain.<po/mo>}:
    Portable Object files and Machine Object files containing the
    translated strings. \texttt{<locale>} is a locale in
    ISO/IEC 15897 format. Currently supported locales are the \texttt{fr}
    (unlocalized French), \texttt{de} (unlocalized German) and
    \texttt{el} (unlocalized Greek) locales.
  \end{itemize}

\item Unit test files

  The \texttt{test\_<module>.py} files are test files written using pytest.
  \texttt{<module>} is the corresponding module described above,
  with tests for the \texttt{Parser} class split between
  \texttt{test\_parse.py} for syntactic checks and
  \texttt{test\_semantic\_parse.py} for semantic checks.


\item Example files

  The \texttt{examples/} directory contains 4 examples of valid
  circuit definition files (with the \texttt{.circuit} extension).
  The circuits are also represented in circuit diagrams in PNG format
  for reference. These files are used for testing, and the user
  may use them as reference when writing their own circuit files.

\item Invalid examples files

  The \texttt{bad\_examples/} directory contains invalid circuit
  definition files. These are used for both manual testing and
  the unit tests mentioned above.
\end{itemize}
