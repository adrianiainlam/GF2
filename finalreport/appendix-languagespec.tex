\section{Specification of the logic description language}
  The logic description language is described by EBNF definition in
  \lstlistingname~\ref{lst:ebnf}. Only single-line comments are allowed
  in the circuit files. The `\texttt{\#}' symbol (ASCII 35) denotes the
  start of a comment, and the new line character (ASCII 10) (or the end of
  file) terminates it. This comment format was chosen since it is
  the easiest to implement (comment formats with opening and closing
  symbols may cause problems with nested comments, among others), and
  it is also a widely employed scheme in many scripting languages
  (e.g.~Bash, Python) and config files (e.g.~Postfix, Nginx).
  By using \texttt{\#} instead of other symbols, there would also
  be built-in support for starting Logic Simulator with a
  shebang.

  It is noted that since a high flexibility was incorporated into the
  grammar by specifying device types as a string of capital letters
  (instead of a list of string literals of
  recognized devices), there was no need to change the grammar
  (hence no need to change the parser) when additional devices were
  added during the maintenance phase. The rationale behind this
  design decision was to allow a potential enhancement of the
  project, such that users can define their own devices, similar
  to how custom classes can be defined in e.g.~C++ and then used
  like a type name.

  \lstinputlisting
      [caption={EBNF definition of the logic description language},
       label={lst:ebnf}]
      {\gitrepopath/finalreport/definition_syntax.txt}
